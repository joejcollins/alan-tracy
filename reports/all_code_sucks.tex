\documentclass[bigger]{beamer}
% \setbeameroption{show only notes}
\setbeamercolor{note page}{bg=white}
\setbeamercolor{note title}{bg=white}

\setbeamersize{text margin left=1cm, text margin right=1cm}
\setbeamertemplate{navigation symbols}{}
\setbeamertemplate{itemize items}{\textbullet}

\usepackage{tikz}
\usepackage{graphicx}
\usepackage{helvet}
\renewcommand{\familydefault}{\sfdefault}
\setbeamerfont{frametitle}{size=\LARGE}
\setbeamerfont{framesubtitle}{size=\Large}
\setbeamerfont{note page}{size=\footnotesize}

\setbeamertemplate{footline}{
  \leavevmode%
  \hfill
  \includegraphics[height=0.5cm]{images/zengenti.png}
  \hspace{1em}
  \vspace{1em}
}

% \usebackgroundtemplate%
% {%
%   \includegraphics[width=\paperwidth,height=\paperheight]{images/test_card.jpg}%
% }

\begin{document}
\title{All Code Sucks}
\subtitle{Why Bad Code is Everywhere and What to Do About It}
\author{Joe J Collins \\
  \href{mailto:j.collins@zengenti.com}{j.collins@zengenti.com} \\
  \href{https://linkedin.com/in/joejcollins}{linkedin.com/in/joejcollins}
}
\institute{\includegraphics[height=1.5cm]{images/zengenti.png}}
\date{21 March 2025}

\begin{frame}[plain]
  \titlepage
  \note{Thanks for taking the time to come out tonight.\\
    I'm Big Joe from Zengenti.\\
    We are a small company in Shropshire. About 70 nerds.\\
    We do websites for universities and local authorities.\\
    I actually don't do any websites, I work in the hosting team.\\
    We maintain a private cloud to run the websites.\\
    We use a combination of Ansible and Python\\
    maintaining about 3000 servers.\\
    But, tonight Matthew,\\
    I am going to talk about, \textbf{why all code sucks}.\\
    \vfil
    We all know good code,\\
    or at least we think we do.\\
    \vfil
    But I should probably define what I mean by sucky code.\\
  }
\end{frame}

\begin{frame}{What is sucky code?}
  \framesubtitle{}
  \begin{quote}
    ``Programs are meant to be read by humans and
    only incidentally for computers to execute.''\\
    \hfill --- Donald Knuth
  \end{quote}
  % \bigskip
  % \begin{quote}
  %   ``It is better to have clean code that doesn't work
  %   than crap code that does.''\\
  %   \hfill --- Robert C. Martin
  % \end{quote}
  \note{My short answer is \dots \textbf{sucky code is hard to read}.\\
    No need take my word for it.\\
    Donald Knuth, the Yoda of Computer Science,\\
    says that code is for humans to read\\
    and sometimes for computers to run.\\
    He is all about the readability.\\
    % \vfil
    % Uncle Bob Martin is more emphatic.\\
    % He uses the term clean code.\\
    % as a proxy for readability.\\
    % He's says that readability is more important than working code.\\
    % If you can understand it, then you can fix it,\\
    % but if you can't understand it and it breaks, you can't fix it.
  }
\end{frame}

\begin{frame}{The Great Hunt for Non Sucky Code}
  \includegraphics[width=\textwidth]{images/companies.png}
  \note{For about about 25 years now,\\
    I have been looking for code that doesn't suck.\\
    And trying to produce code that doesn't suck.\\
    I've worked in companies large and small.\\
    But pretty much, all the code sucked.\\
    \vfil
    This does beg the question, why does all code suck?\\
  }
\end{frame}

\begin{frame}{Why Code Sucks}
  \framesubtitle{The Statistics Favour Suck}
  \begin{itemize}
    \item Half of everything is below average
      \pause
    \item Sturgeon's Law
      \pause
    \item The 3 Year Old Programmer
  \end{itemize}
  \note<1>{
    On the whole I think the odds are against us.\\
    Straight out of the gate,\\
    half of everything is going to be below average.\\
    Well, below the median.\\
    That's just the way averages work.\\
  }
  \note<2>{Then there is Sturgeon's Law.\\
    Sturgeon was explaining why most science fiction is low quality.\\
    And came up with the pithy answer,\\
    ``90\% of everything is shit''.\\
    The observation works here too.\\
  }
  \note<3>{
    Then there is our experience.\\
    When I started out\\
    the average programmers experience was 3 years.\\
    Now 25 years later that hasn't changed.\\
    Probably because demand outstrips supply.\\
    Which has been nice.\\
    \vfil
    So if the odds are against us\\
    maybe the organizations we work for, will help.\\
    Or perhaps not.\\
  }
\end{frame}

\begin{frame}{Why Code Sucks}
  \framesubtitle{Organisations Tend To Suck}
  \begin{itemize}
    \item Software Startups
      \pause
    \item Summer Student Projects
      \pause
    \item Prototypes in Production
      %   \pause
      % \item The New Project Effect
      %   \pause
      % \item The Agile Manifesto
  \end{itemize}
  \note<1>{
    The romantic image of a software startup is\\
    a couple of guys in a garage.\\
    I have actually see this quite a bit.\\
    For most start ups the two guys are the dad and the son.\\
    The dad is the salesman.\\
    And the son is the programmer.\\
    They are inherently under resourced.\\
    Writing readable code is a luxury they can't afford.\\
  }
  \note<2>{
    The other kind of startup I have seen\\
    occurs in big companies.\\
    The summer student project.\\
    Alternatively called the \textbf{unsupervised use of new technology}.\\
    All the experienced programmers are on holiday or busy.\\
    So they give the new technology to the summer students,\\
    who give it a go.\\
    And if it runs they put it into production.\\
  }
  \note<3>{
    This last point is also a general point.\\
    Any software that appears to work goes into production.\\
    Not because anyone thinks it's a good idea\\
    but because there is a commercial imperative.\\
    Having learnt from the prototype,\\
    the plan was to throw it away and build it properly.\\
    But that never happens.\\
    It is always put into production.\\
    \vfil
    Just in case you haven't lost all hope\\
    I would say that suck\\
    is actually built into human psychology.\\
  }
  % \note<4>{
  %   So,\\
  %   say you start a new project,\\
  %   well resourced with all the best intentions.\\
  %   The organization is still against you.\\
  %   Who is available to work on the project?\\
  %   Anyone who is new or is anyone whose on the bench.\\
  %   This isn't necessarily a bad thing.\\
  %   The average age of the team that\\
  %   helped put Neil Armstrong to the moon was 28.\\
  %   But they were all new graduate engineers.\\
  %   The experienced engineers all had jobs\\
  %   and why take a risk on a new project?
  % }
  % \note<5>{
  %   Then there is \textbf{Agile Management Practice} \ldots\\
  %   it was such a great idea.\\
  %   That we should respond to change over following the plan.\\
  %   It didn't say we shouldn't plan.\\
  %   But I was there 25 years ago,\\
  %   I was there the day the strength of Men failed\\
  %   and we deliberately misread the Agile Manifesto\\
  %   as `no need plan'.\\
  %   Just start programming!
  %   \vfil
  %   And as if this wasn't bad enough,\\
  %   just to make you feel completely hopeless,\\
  %   I would say that suck\\
  %   is actually built into human psychology.
  % }
\end{frame}

\begin{frame}{Why Code Sucks}
  \framesubtitle{The Psychology of Suck}
  \begin{itemize}
    \item The Illusion of Explanatory Depth
      \pause
      % \item The Lake Wobegon Effect
      %   \pause
    \item Availability Bias
  \end{itemize}
  \note<1>{
    We are unaware of our ignorance.\\
    Because we are ignorant of it.\\
    We don't know what we don't know.\\
    But writing code is about explaining things in detail.\\
    So it is no surprise that we struggle.\\
  }
  \note<2>{
    A couple of years ago we used a static analysis tool on our code.\\
    A gadget called `CodeScene' for anyone that is interested.\\
    To my surprise it said most of our code was good.\\
    But it did point to the five worst files.\\
    Those are the files I spend most of my time working on.\\
    So those are the bits I remember.\\
    It turns out all code doesn't actually suck,\\
    I just feels like it.
    \vfil
    Nevertheless, if that is what you are working on,\\
    makes sense to tool up for sucky code.\\
  }
\end{frame}

\begin{frame}{What To Do}
  \framesubtitle{Adapted from Michael Feathers}
    \begin{itemize}
      \item \textbf{Discovery} Log behavior
      \item \textbf{Exploration} Scripted investigation
      \item \textbf{Automating} Test harnesses (or clamps)
      \item \textbf{Deconstruction} Separate the target code
      \item \textbf{Enable} Switching with a feature flag
      \item \textbf{New Code} Side by side rewrite
    \end{itemize}
  \note<1>{
    This is my adapted 6 step plan to work on sucky code.\\
    An initial discovery phase is to find out how the code is actually used.\\
    Then we can explore the code and find it's quirks.\\
    Our intermediate goal is to write some automated tests.\\
    To protect us from making changes that might break things.\\
    With the tests in place we can start to deconstruct the code.\\
    And target the bits that need attention.\\
    Using a feature flag will enable us to switch between\\
    old and new code.\\
    Finally we can get on with the job of writing new code.\\
    \vfil
    I have skipped the step about\\
    getting control of the development environment.\\
    We use ephemeral environments at Zengenti to keep things clean.\\
    But that is a whole other talk, for another day.\\
  }
  \note<2>{
    As a demo I have an API with one end point.\\
    You can enjoy the squiggles, however, it does run.\\
    \textbf{make app}\\
    Let's take a look and see what it does.\\
    \textbf{Open http://127.0.0.1:8000/docs}\\
    \textbf{Try 2 and 42}\\
    I can't really tell what is going on.\\
    We really need to discover how the end point is being used.\\
    So adding some logging is a good idea.\\
    \textbf{demo01log}\\
    Michael Feathers says don't change any thing until you have tests,\\
    but he admits that sometimes you have to add stuff like logging.\\
    \textbf{Try 2, 4 and 42}\\
    We have a log of actual user behavior.\\
  }
  \note<3>{
    Now we know how the end point is used.\\
    We can start to explore the code a bit more.\\
    \textbf{Open tests/app.http}\\
    We can use http scripts to mimic the UI behavior.\\
    \textbf{demo2http01}\\
    There's a Jetbrains client for this or in this case a VSCode extension.\\
    But you could equally well use a Jupyter notebook.\\
    \textbf{demo2http02}\\
    This allows us to do a more systematic investigation.\\
    Looks like a string is coming back.\\
    Not a number.\\
    \textbf{Fix inverted commas}\\
    We can get a feel for the behavior and quirks.\\
    We might even start to get an insight into the author's intent.\\
  }
  \note<4>{
    With this knowledge we can write some proper automated tests.\\
    The kind of tests we could use in a CI/CD pipeline.\\
    \textbf{Open tests/app\_test.py}\\
    These are the clamps that will hold the code in place.\\
    As Michael Feathers calls them.\\
    We run these tests every time we make a change.\\
    To make sure we haven't broken anything.\\
    \textbf{demo3test01}\\
    \textbf{demo3test02}\\
    This will give us the confidence to make changes.\\
    Every time we make a change we run the tests.\\
    \textbf{make test}\\
    We get a little report each time.\\
    To let is know it is alright to continue.\\
  }
  \note<5>{
    With the automated tests in place.\\
    We are ready to start deconstructing the code.\\
    \textbf{Open app.py}\\
    We can start to separate the target code.\\
    \textbf{demo4function01}\\
    Since we don't really know what the code does.\\
    We can't give it a meaningful name.\\
    Michael Feathers suggests just mashing the keyboard.\\
    We can check that it works in all it's ugliness.\\
    By rerunning the tests.\\
    \textbf{make test}\\
  }
  \note<6>{
    Stay with me we are getting close.\\
    Now we can enable a switch to control any new code.\\
    There are lots of tools to do this.\\
    We favour Unleash because it comes for free with Gitlab.\\
    \textbf{Open /joejcollins/alan-tracy/-/feature\_flags}\\
    Here I have a feature flag to turn a function on and off.\\
    I can use this to switch between the old and the new code.\\
    \textbf{demo5flag01}\\
    \textbf{make test}\\
    So lets rewrite the code.\\
    \textbf{demo6new01}\\
    This is actually a cut and paste from ChatGPT.\\
    It is still a bit of a mess, but I am running out of time.\\
    Things still work.\\
    The code still sucks, but it sucks less.\\
  }
\end{frame}

\begin{frame}{The Challenge}
  \framesubtitle{Where I need your help}
  \begin{quote}
    ``It's not that hard...''\\
    \hfill --- Billy Beane
  \end{quote}
  \bigskip
  \begin{quote}
    ``It's incredibly hard''\\
    \hfill --- Ron Washington
  \end{quote}
  \bigskip
  \note{To quote Moneyball,\\
    this challenge is both easy and difficult.\\
    The example was one function and one file.\\
    But the code we are working with has 3500 files.\\
    Some is hard to read,\\
    some has been improved and some it half way between.\\
    But it often feels like we have three different code bases.\\
    All jumbled up.\\
    \vfil
    It turns out the improving code is easy\\
    but managing the process is hard.\\
    Knowing which bits to improve really hard.\\
    I have no answer to that, I am kind of hoping you do.\\
  }
\end{frame}

\begin{frame}[plain]
  \centering
  {\Huge \bfseries Thank You!}\\[1cm]
  {\large Joe J Collins} \\[0.5cm]
  \includegraphics[height=1.5cm]{images/zengenti.png}
  \note{Thank you for listening.\\
    I'm Big Joe from Zengenti.\\
    If you thought this was interesting,\\
    and would like to work with us,\\
    please get in touch.\\
    If you have other ideas how to work with sucky code,\\
    I'm all ears.\\
    Or if you have an idea how to manage the process,\\
    Let's talk.\\
  }
\end{frame}

\end{document}
